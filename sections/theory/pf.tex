To simulate the grid a way to solve the powerflow problem is needed. To solve powerflow we assume,
that we know all the production and consumption of all prosumers, as well as the voltage at the so-called slack
node (for this node the consumed/produced power is not known). Using a powerflow solver we can then determine the voltage
at all the other nodes in the grid as well as the power at the slack node. This information then also suffices to determine
the currents through all cables. Taken together these quantities give a fairly good overview over how the grid performs.

\begin{center}
    \begin{tabular}{ c c c c }
    Node type          & Power     & Voltage & Count \\ 
    \hline
    \textbf{Slack}     & Unknown   & Known   & 1     \\  
    \textbf{Prosumer}  & Known     & Unknown & n    
    \end{tabular}
\end{center}



To formulate the powerflow equation, we can start form the general assumption
that the current entering node has to equal the current exiting it.\\
There is two sources for current through a node:

\begin{itemize}
    \item Current due to the prosumer at that node
    \item Current from or to other nodes
\end{itemize}

Defining the current due to the node itself is straight forward by considering
the complex power equation (eq: \ref{eq:ac:complex}):

\begin{equation}
    \begin{aligned}
        S_i     &= V_iI_i^*\\
        I_i     &= \frac{S_i^*}{V_i^*}
    \end{aligned}
    \label{eq:pf:current_due_to_prosumer}
\end{equation}

Defining the current due to the other nodes requires the use of Ohm's law (eq. \ref{eq:ac:ohm_complex}).
The current flowing from the current node (node $i$)
to one of the adjacent nodes (node $j$) will be a result of the
voltage difference between the current node and its neighbour:

\begin{equation}
    \begin{aligned}
        \Delta V_{ij} &= Z_{ij} I_{ij} \\
        I_{ij}        &= (V_i - V_j) y_{ij}
    \end{aligned}
\end{equation}

Where $y_ij$ is the admittance between nodes $i$ and $j$.
Summing up all currents from all the neighbours (we can ignore the case $i = j$ as $\Delta V_{ii} = 0$)

\begin{equation}
    \begin{aligned}
        I_{i} &= \sum_{i \ne j}^N (V_i - V_j) y_{ij}\\
              &= V_i \sum_{i \ne j}^N y_{ij} - \sum_{i \ne j}^N V_j y_{ij}
    \end{aligned}
\end{equation}

Finally, equating the two sources for current we get to the formulation of the power flow
equation:

\begin{equation}
    \begin{aligned}
        V_i \sum_{i \ne j}^N y_{ij} - \sum_{i \ne j}^N V_j y_{ij} = \frac{S_i^*}{V_i^*}
    \end{aligned}
\end{equation}

This powerflow equation is non-linear, which thus necessities the use of a computational
solver. Two solvers, their advantages and disadvantages as well as implementation details will
be discussed subsequently.

