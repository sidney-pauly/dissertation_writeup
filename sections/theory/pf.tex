To formulate the power flow equation, we can start form the general assumption
that the current entering a node has to equal the current exiting it, i.e.
there is conservation of charges.\\
There are two sources for current through a node:

\begin{itemize}
    \item Current due to production/consumption at the node
    \item Current from or to other nodes
\end{itemize}

The current due to production/consumption at node $i$ itself is
obtained by considering
the complex power \autoref{eq:ac:complex}:

\begin{equation}
    \begin{aligned}
        S_i     &= V_iI_i^*\\
        I_i     &= \frac{S_i^*}{V_i^*}.
    \end{aligned}
    \label{eq:pf:current_due_to_prosumer}
\end{equation}

Defining the current due to the other nodes requires the use of
Ohm's law (\autoref{eq:ac:ohm_complex}).
The current flowing from the current node (node $i$)
to one of the adjacent nodes (node $j$) will be a result of the
voltage difference between the current node and its neighbour:

\begin{equation}
    \begin{aligned}
        \Delta V_{ij} &= Z_{ij} I_{ij} \\
        I_{ij}        &= (V_i - V_j) Y_{ij},
    \end{aligned}
    \label{eq:pf:neighbour_current}
\end{equation}

where $Y_{ij}$ is the admittance between nodes $i$ and $j$.
Summing over \autoref{eq:pf:neighbour_current} for all neighbours of $i$:

\begin{equation}
    \begin{aligned}
        I_{i} &= \sum_{i \ne j}^N (V_i - V_j) y_{ij}\\
              &= V_i \sum_{i \ne j}^N y_{ij} - \sum_{i \ne j}^N V_j y_{ij}.
        \label{eq:pf:current_at_node}
    \end{aligned}
\end{equation}

The case $i = j$ can be ignored as $\Delta V_{ii} = 0$.\\
\\
Equating the two sources for current \autoref{eq:pf:current_due_to_prosumer}
and \autoref{eq:pf:current_at_node} results in the power flow equation:

\begin{equation}
    \begin{aligned}
        V_i \sum_{i \ne j}^N Y_{ij} - \sum_{i \ne j}^N V_j Y_{ij} = \frac{S_i^*}{V_i^*}.
        \label{eq:pf:pf_eq_large}
    \end{aligned}
\end{equation}

One further simplification can be employed by using an admittance matrix.
The elements of which are:

\begin{equation}
    y_{ij} =
    \begin{cases}
        \sum_{i \ne j} y_{ij} & \text{for} \ i = j\\
        -Y_{ij}               & \text{for} \ i \ne j.
    \end{cases}
    \label{eq:pf:admittance_matrix}
\end{equation}

Thus equation \ref{eq:pf:pf_eq_large} can be further simplified:

\begin{equation}
    \sum_{j=1}^N V_j y_{ij} = \frac{S_i^*}{V_i^*}.
    \label{eq:pf:full_pf_eq}
\end{equation}

Note the change from using admittance $Y_{ij}$ to the admittance matrix
elements $y_{ij}$.\\
\\
This power flow equation is non-linear, which thus necessities the use of a computational
solver. Two such solvers, are introduced in \autoref{sec:methods:pf}\\
\\
To simulate the grid a way to solve the power flow problem is needed. 
To solve power flow we assume,
that we know all the production and consumption for all prosumers,
as well as the voltage at the so-called slack
node (for this node the consumed/produced power is not known). Using a
power flow solver we can then determine the voltage
at all the other nodes in the grid as well as the power at the slack 
node. This information then also suffices to determine
the currents through all cables. Taken together these quantities 
give a fairly good overview of how the grid performs.

\begin{figure}[H]
    \centering
    \begin{tabular}{ l|r r r }
        Node type          & Power     & Voltage & Count \\ 
        \midrule
        \textbf{Slack}     & Unknown   & Known   & 1     \\  
        \textbf{Prosumer}  & Known     & Unknown & n    
    \end{tabular}
    \caption{
        Types of nodes relevant to solving power flow.
    }
\end{figure}
