The electrical grid can be represented as a graph. 
In a graph theory a graph is defined as a set of nodes and edges. A node
is some point in the graph, whilst an edge is a connecting element between exactly
two nodes.\\
\\

The simplest form of a graph is where the nods and edges do not have
any additional information attached to them and the edges are bidirectional.
I.e. the edge $A \to B$ is the same as the edge $B \to A$. Even though the setup
of such a graph seems simple, there are numerous metrics and complicated questions
that can be asked at this level. For example trying to find the quickest way between
two points in a graph is already a non-trivial problem that has its own class of algorithms
(e.g. A*, etc.). In the context of this project graph-theory becomes important in two ways.
Firstly it informs how the grid data is represented and processed and secondly it can
be used to classify certain configurations of the grid, through graph measures.

\subsection{Grid data representation}

In the most fundamental terms the grid can be represented as nodes, which are the 
producers or consumers (also called prosumers) and cables, which are the edges.
The prosumers have a production or consumption capacity $S_i$ (complex power see eq. \ref{eq:ac:complex}).
The cables have an admittance $Y_{ij}$ (see eq. \ref{eq:ac:admittance}).\\
\\
3 ways to represent the graph data will be subsequently used. 
Each records the full information of the graph, but
has different advantages and disadvantages. Note that the examples shown below all record
the admittance of each cable in addition to the info that the cable exists. Sometimes
this info is not need in which case it can be omitted and the representation further simplified/

\subsubsection{Edge list}

In edge list representation there is an entry for each edge in the graph in form of a vector:

\begin{equation}    
    \mathfrak{E_{ij}} = (i,\ j,\ Y_{ij})
\end{equation}

here $i$ and $j$ are the index of node A and B connected by the edge and
$Y_{ij}$ the admittance of the edge (cable).
For sparse graphs (graphs where only a handful of all possible edges $\mathfrak{E_{ij}}$ exist), this method
of storing the graph data is very efficient as only as many entries as cables are required. The disadvantage
algorithmically is that finding a specific connection between is not so efficient as it has to be searched for amongst
all entries. E.g. to look up admittance between nodes a and b that specific edge has to be found.

\subsubsection{Matrix form}

In matrix form each row and column represents a node. Each entry in the matrix is the admittance between nodes $i$ and $j$.
If the admittance is not to be recorded than if a node exists the entry can be set to 1 and if not to 0 instead:

\begin{equation}
    Y =
    \begin{pmatrix}
        Y_{11} & Y_{12} & ... & Y_{1n}\\
        Y_{21} & ...    & ... & ...   \\
        ...    & ...    & ... & ...   \\
        Y_{n1} & ...    & ... & Y_{nn}\\
    \end{pmatrix}
\end{equation}

As the cables in a grid are symmetrical (admittance is the same for current from a to b as it is form b to a), 
the matrix is symmetrical as well: $Y_{ij} = Y_{ji}$. Further in pure graph representation all diagonal elements are
zero as there are no cables running in a loop starting and ending at the same node.\\ 
One big advantage of this representation is that it allows for quick lookups, i.e. if one needs the admittance between nodes a and b
one can simply look up that entry in the matrix.\\
The disadvantage of this form of representation is that it is very space inefficient, as it requires
$N^2$ entries (where $N$ is the number of nodes). Especially for sparse graphs a lot of these entreis contain
zeros and are thus redundant. This drawback in memory and iteration efficiency
can be mitigated by clever data structure arrangement, which will be discussed later. 

\subsubsection{Node representation}

In the node representation the graph is represented by the connections
each node has and the admittance to each other node:

\begin{equation}
    \mathfrak{N_i} = \sum_{i \ne j, Y_{ij} > 0} (j, \ Y_{ij}) 
\end{equation}

This representation is especially useful for things like flooding algorithms where it is
relevant to know which nodes can be reached from any given node in question. 

\subsection{Graph measures}