In an AC system both current and voltage oscillate. This can be described using:

\begin{equation}
    \begin{split}
    V(t) = V_m cos(\omega t + \theta_V)\\
    I(t) = I_m cos(\omega t + \theta_I),
    \end{split}
\label{eq:ac:ac}
\end{equation}

where $V_m$ and $I_m$ are the magnitudes of voltage and current,
$\omega$ the angular velocity of the oscillations, and $\theta_V$ and $\theta_I$
the phase offset of voltage and current.\\
Power as a function of time can then be defined as:

\begin{equation}
    P(t) = V(t)I(t)
    \label{eq:ac:power_t}
\end{equation}

Using trigonometric identities, the substations of \autoref{eq:ac:ac} in \autoref{eq:ac:power_t} hold:

\begin{equation}
    \begin{aligned}
        P(t) = \frac{1}{2} V_m I_m (\cos (\theta_V-\theta_I) 
        + \cos(2(\omega t + \theta_V))\cos(\theta_V-\theta_I)
        + \sin(2(\omega t + \theta_V))\sin(\theta_V-\theta_I)),
    \end{aligned}
    \label{eq:ac:power_ac_long}
\end{equation}

which can be simplified using

\begin{equation}
    \begin{aligned}
        \theta  &= \theta_V - \theta_I \\
        |V|     &= \frac{V_m}{\sqrt{2}} \\
        |I|     &= \frac{I_m}{\sqrt{2}},
    \end{aligned}
    \label{eq:ac:power_def}
\end{equation}

where $\theta$ is the impedance angle and $|V|$ and $|I|$ are the root mean squares.
This allows to define power as
         
\begin{equation}
    \begin{aligned}
        P(t)   &= P_R(t) + P_X(t)\\
        P_R(t) &= |V||I| \cos(\theta) (1 + \cos(2(\omega t+\theta_V)))\\
        P_X(t) &= |V||I| \sin(\theta) (1 + \sin(2(\omega t+\theta_V))).
    \end{aligned}
    \label{eq:ac:power_react_and_capacitive}
\end{equation}

$P_R$ is called the real power, and represents power used in resistive loads. $P_X$
is reactive power, which oscillates within the grid, repeatedly charging and discharging
capacitive loads, but not being consumed. \\
If we are interested in steady state scenarios, then we can assume $\omega$ and $\theta_V$ 
to be constants. Also, when averaging over long periods of time
$\sin(A t + B)$ and $\cos(A t + B)$ are 0.\\ This means that the steady-state,
temporal average of the real and reactive powers are:

\begin{equation}
    \begin{aligned}
        P_R(t) &= P = |V||I| \cos(\theta)\\
        P_X(t) &= Q = |V||I| \sin(\theta)\\
    \end{aligned}
    \label{eq:ac:power_react_and_imag}
\end{equation}

It is handy to define these as the real and imaginary parts of a complex number:

\begin{equation}
    \begin{aligned}
        S     &= V I^*\\
        S     &= P + iQ\\
        Re(S) &= P\\ 
        Im(S) &= Q.
    \end{aligned}
    \label{eq:ac:complex}
\end{equation}

\begin{figure}[H]
    \centering
    \begin{tikzpicture}[scale=2]
    \draw[->, line width=.7mm] (-1, 0) -- (1, 1.5) node[pos=.65](s){};
    \draw[->, line width=.7mm] (-1, 0) -- (1, 0) node[pos=.5](p){}; 
    \draw[->, line width=.7mm] (1, 0) -- (1, 1.4) node[pos=.5](q){};  

    \draw[line width=.7mm] (0, 0) arc (0:45:.8);

    \node[text width=2cm] at (s)[above] {\large $S$};
    \node at ($(q) + (.1, 0)$)[right] {\large $Q$};
    \node at ($(p) + (0, -.1)$)[below] {\large $P$};
    \node at (-.3, .25) {\large $\theta$};
\end{tikzpicture}
    \caption{
        The power triangle showing the relation between complex power $S$,
        real power $P$, reactive power $Q$ and the power angle $\theta$.
    }
    \label{fig:ac:power_triangle}
\end{figure}

To describe an AC grid we also need to introduce a complex version
of Ohm's law:

\begin{equation}
    V = ZI,
    \label{eq:ac:ohm_complex}
\end{equation}

where $Z$ is impedance, which is defined as:

\begin{equation}
    Z = R + iX,
    \label{eq:ac:impedance}
\end{equation}

where $R$ is the resistance and $X$ is the reactance. Reactance measures how much
energy will be stored in the cable within a current cycle due to a magnetic
field being created\\
\\
Lastly, we can also define an AC-equivalent for conductance,
which is called admittance, $Y$:

\begin{equation}
    Y = 1/Z
    \label{eq:ac:admittance}
\end{equation}