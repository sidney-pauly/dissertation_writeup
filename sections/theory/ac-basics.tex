To the grid operates on alternating current, thus an
understanding of the underlying formulas governing ac systems will be required to analyse it. \\
In an AC system both current and voltage oscillate. This can be generally described using the following two equations:

\begin{equation}
    V(t) = V_m cos(\omega t + \theta_V)
    \label{eq:ac:voltage_ac}
\end{equation}

\begin{equation}
    I(t) = I_m cos(\omega t + \theta_I)
    \label{eq:ac:current_ac}
\end{equation}

where $V_m$ and $V_m$ are the magnitudes of voltage and current,
$\omega$ the angular velocity of the oscillations and $\theta_V$ and $\theta_I$
the phase offset of Voltage and current.\\
Power as a function of time can then be defined as:

\begin{equation}
    P(t) = V(t)I(t)
    \label{eq:ac:power_t}
\end{equation}

Expanding this leads to the following expression:

\begin{equation}
    \begin{aligned}
        P(t) = \frac{1}{2} V_m I_m (\cos (\theta_V-\theta_I) 
        + \cos(2(\omega t + \theta_V))\cos(\theta_V-\theta_I)
        + \sin(2(\omega t + \theta_V))\sin(\theta_V-\theta_I))
    \end{aligned}
    \label{eq:ac:power_ac_long}
\end{equation}

which can be simplified using

\begin{equation}
    \begin{aligned}
        &\text{Impedance angle}: \quad  \theta &= \theta_V - \theta_I \\
        &\text{Root mean square of } V: \quad  |V| &= \frac{V_m}{\sqrt{2}} \\
        &\text{Root mean square of } I: \quad  |i| &= \frac{I_m}{\sqrt{2}} \\
    \end{aligned}
    \label{eq:ac:power_def}
\end{equation}
         
\begin{equation}
    \begin{aligned}
        P(t)   &= P_R(t) + P_X(t)\\
        P_R(t) &= |V||I| \cos(\theta) (1 + \cos(2(\omega t+\theta_V)))\\
        P_X(t) &= |V||I| \sin(\theta) (1 + \sin(2(\omega t+\theta_V)))\\
    \end{aligned}
    \label{eq:ac:power_react_and_capacitive}
\end{equation}

If we are interested in steady state scenarios, then we can assume $\omega$ and $\theta_V$ 
to be constants. Over longer periods of time the average of $\sin(A t + B)$ and $\cos(A t + B)$ are 0.\\
We can thus simplify $P_R$ and $P_X$ further:

\begin{equation}
    \begin{aligned}
        P_R(t) &= P = |V||I| \cos(\theta)\\
        P_X(t) &= Q = |V||I| \sin(\theta)\\
    \end{aligned}
    \label{eq:ac:power_react_and_imag}
\end{equation}

It is handy to define these as the real and imaginary parts of a complex number:

\begin{equation}
    \begin{aligned}
        S     &=  V I^*\\
        S     &= P - iQ\\
        Re(S) &= P\\ 
        Im(S) &= Q
    \end{aligned}
    \label{eq:ac:complex}
\end{equation}

To describe an AC grid we also need to introduce a complex version
of Ohm's law which reads like this:

\begin{equation}
    V = ZI
    \label{eq:ac:ohm_complex}
\end{equation}

where $Z$ is impedance which is defined as:

\begin{equation}
    Z = R + iX
    \label{eq:ac:impedance}
\end{equation}

where $R$ is resistance and $X$ is reactance. Reactance measures how much
energy will be stored in the cable within a current cycle due to a magnetic
field being created\\

Lastly we can also define an ac-equivalent for conductance, which is called admittance $Y$:

\begin{equation}
    Y = 1/Z
    \label{eq:ac:admittance}
\end{equation}