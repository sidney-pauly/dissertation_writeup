% Applied to venios customers
% As such done on real grids, instead of sythetic examples.
% Specifically this is being done on entire grid areas, thus isntead of disconnecitng
% lines we start from the existing grid structure and reconfigure it for better results
% No other load shedding mechanisms, or other control mechanisms
% Fixed generation and consumption (this si largly due to regulations)
% This is to be applied in an operational context: i.e. we take the current state
% of the grid with its current generation and topology and determine the best switch
% state within. Defining the "best" switchstate is not straightforward as switching
% impacts different parameters in the gird in different ways. Thus different
% optimization goals are presented, which can then be selected as needed or prefered
% in the operational context

% One restriction being imposed witin this work is single feed in, meaning that only
% one transformer supplies a given distribution grid. 

As a study case the grid of one of Venios' Swiss customers will be used. 
They operate over a thousand
low voltage grids in rural, suburban and urban areas\autocite{venios}. 
A grid in this context is a gird area containing a transformer and supplying
a neighbourhood of 60 or more buildings\autocite{venios}. These grids have switches
within and switches connecting them to neighbouring grids. Within the standard
switch state (SSS), all switches to other grids are open. This switch state is
used under normal operating conditions\autocite{venios}. To test the effect of switching
we select geographical boxes within the grid operators' region containing 5-10 grids each.
We then examine different switching states and apply different optimizers with the goal
of finding switching states with improved operational parameters over the SSS.\\
\\
The laws in Switzerland mandate equal and continuous access to the grid for every household.
Further Switzerland allows feed-in by private households with a right to sell guaranteed
for small producers\autocite{venios}. The implications of this are that continuous supply
and continuous feed-in has to be guaranteed at the distribution grid level with penalties
imposed in case of violation. Subsequently, any disconnection of grid members needs
to be avoided and mandated load shedding is both costly and needs to be distributed
over grid customers equally\autocite{venios}. All of this combined presents a strong
argument for switch state optimization as it poses an opportunity to improve grid operational
parameters with almost no costs to the grid operator.