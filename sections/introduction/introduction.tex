The electrical grid is divided into multiple levels. Whilst the higher levels of the grid connect
the entire countries the lower level grids go from house to house and are hence also called 
distribution grids. In the past electricity was mostly produced at higher levels in power plants
and consumed at the lower level. The load patterns in the lower levels of the grid where
mostly predictable as they closely followed peoples work days. All this meant that a simpler setup
in the these girds was sufficient. In the past 20-30 this has changed. A lot of energy is now
being produced at these lower levels, e.g. through solar panels. Further, load patterns have shifted
as well with people charging their electric cars, heating with heat pumps and generally consuming
their energy at different times (e.g. because of home office)\autocite{venios}.
To fix load and instability issues arising
from this, there are broadly two options, either you build more physical infrastructure "blindly" to always
have a safe margin or you gather insight into the grid by measuring and simulating laods. This can either
be used in real time to actively control the grid or beforehand to have a better idea of load margins as well
as to more efficiently add new physical infrastructure. One active control measure that can be taken
to optimize load is by switching switches in the grid controlling how much of it is interconnected.
Generally a distribution grid connects up a few houses to a few streets depending on the region or country.
Each of these "grid islands" is connected up to the higher level through a transformer. By changing over
switches in the grid one can connect up these islands, making bigger grid areas or regroup certain parts
of these islands into neighbouring ones. Both can have effects on the grid in a non-linear complex way.\\

