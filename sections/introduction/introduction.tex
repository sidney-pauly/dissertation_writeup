The electrical grid is divided into multiple levels. 
Whilst the higher levels of the grid connect
entire countries the lower level grids go from house
to house and are hence also called 
distribution grids. In the past electricity
was mostly produced at higher grid levels in power plants
and consumed at the lower levels. The load patterns in
the lower levels of the grid where
mostly predictable as they closely followed peoples
work days. All this meant that a simpler setup
in the these girds was sufficient. In the past 20-30
this has changed. A lot of energy is now
being produced at these lower levels, e.g. through solar
panels. Additionally, load patterns have also shifted
with the introduction of electric cars,
heat pumps and home office\autocite{venios}.\\
\\
There are broadly two approaches that can be taken to solve congestion and instability 
issues that arise from these recent developments. The first is to build more
physical infrastructure "blindly" to
ensure a safe operating margin for any grid state that is conceivable. 
The alternative is to gain insight into the grids' operation through measurements,
predictions and simulations, in industry this is called building a "digital twin".
Insights gained through this process can be both used in planning and operation.\\
During planning they help to get a better idea of what actual safety margin
is required and to thus build physical infrastructure more efficiently.\\
During operation, it can be
used to take active measures, like shutting down prosumers and 
thus preventing critical load conditions.\\
\\
One measure that can be taken, which is both useful during planning and operation, is to
change switch configurations within the grid. By switching switches one
can affect which consumers get supplied by which producers and through what
paths. However, switching a switch has non-local and non-linear effects.
This means it is very hard to predict the effect a changed switch state
will have on the grid. \\
\\
In this work we quantify these effects through the simulation of
different switch states.
We then use this knowledge to develop optimizers that can reliably
find improved switch states.
