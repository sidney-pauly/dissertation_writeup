Within literature two adjacent problems exist: Optimal Transmission Switching (OTS)\autocite{ots_lit_review}
and Switch Allocation Problem (SAP)\autocite{switch_allociation_lit_review}.\\
\\
OTS is usually focused on transmission grids and aims to reduce the overall
operation costs by taking out certain transmission lines. It often also incorporates other grid control
mechanisms into the optimization. A key difference between OTS and the aim of this
work is that OTS generally avoids islanding completely, i.e. it only considers one big interconnected grid 
area. Also, because usually does not deal with the distribution grid, transfomers as the connection point
to the rest of the grid are generally not considered. Lastly within the context of OTS transmition lines are
switched, that means that    \\
\\
SAP is focused on the distribution grid, however instead of asking which switches to switch
it asks where switches should ideally be installed to increase gird reliability and reduce costs.
SAP does not focus on the question of how the switches in the grid should
be configured during operation.\\
\\
Literature on OSSP or OSCP seems to be very limited. The only available
literature we are aware of are the works of
Kerzel et al.\autocite{optimal_switch_configuration} and
Wolter et al.\autocite{meshed_grids}. Both of these also consider the german grids, 
with their relatively high count of producers in the distribution grid. 


