The aim of this dissertation is to explore the effects switching switches has on the grid, by simulating them
and measuring effects on various grid load and stability quantities. In a second step this analysis can then be
used to develop algorithms that find an improved or even optimal switch state.\\
Within literature this problem is known as Optimal Transmission Switching (OTS)\autocite{ots_lit_review}.
The OTS problem has multiple formulations, depending on what parameters are being optimized for and 
in which operational context it is employed.\\
\\

% Applied to venios customers
% As such done on real grids, instead of sythetic examples.
% Specifically this is being done on entire grid areas, thus isntead of disconnecitng
% lines we start from the existing grid structure and reconfigure it for better results
% No other load shedding mechanisms, or other control mechanisms
% Fixed generation and consumption (this si largly due to regulations)
% This is to be applied in an operational context: i.e. we take the current state
% of the grid with its current generation and topology and determine the best switch
% state within. Defining the "best" switchstate is not straightforward as switching
% impacts different parameters in the gird in different ways. Thus different
% optimization goals are presented, which can then be selected as needed or prefered
% in the operational context

% One restriction being imposed witin this work is single feed in, meaning that only
% one transformer supplies a given distribution grid. 
