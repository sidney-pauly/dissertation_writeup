There are two types of measures that can be applied on a template or on the entire grid 
\begin{enumerate}
  \item A priori measures that result from the grid topology and the parameters
  of the grid components
  \item Measures resulting from running powerflow (or other simulations)
\end{enumerate}

The former are quicker to obtain and are generally good to classify the templates while
the later ones are the most relevant to decide how good a template performs.\\
\\
Ideally a correlation between measures of type one and measures of type two can be
established, as that would make it possible to specifically generate templates
maximizing those quantities.
\\
An added complication arises from the fact that measures that differ per template
might not differ between partitions of the grid. In these cases measures need to be introduced
that assess the difference of this measure between the templates. 

\subsection{A priori measures}

\subsubsection{Atomic island count}

The simplest measure has already been introduced, it is just the number
of atomic island per template: $|T_i|$. The number of atomic
islands in the grid stays constant regardless of switch configuration.
Thus, this measure cannot be used as a comparison between different
partitions.
Further the average template size also doesn't change as: 

\begin{equation}
  \frac{\sum_i^{|T|} |T_i|}{|T|} = \frac{n}{|T|}, \quad \forall T
\end{equation}

To compare the between different partitions the variance, quartiles,
min or max can be used instead.

\subsubsection{Component count}

Component count can be more complicated than atomic island count, depending on how
it is carried out. It could be desirable to exclude dead ends within a template.
These can occur if there is an edge running to a switching cabinet with an open switch. 
In that case it might be desirable to exclude all components in the switching cabinet
and the edge leading to it. If components are excluded depending on the switch state 
counts between partitions may be different and averages hold significance.\\
Of particular interest might
also be counting components of a specific type, like prosumers, cables, etc.

\subsubsection{Geographical extend}

The geographical extent of a grid can be measured in different ways. A simple way is to
measure the distance, $d_{max}$ between the furthest nodes in the grid:

\begin{equation}
  d_{max} = max(\{d_{ij} : i, j < n\})
  \label{eq:measures:geo_extend}
\end{equation}

Similar considerations about excluding components apply.

\subsubsection{Net power}

The net power of a template is the sum over all power produced and consumed:

\begin{equation}
  \sum_{i = 0}^n S_i
\end{equation}

Again this number does not change between partitions.

\subsubsection{Deviation from standard switch state}

There are two main ways to measure this quantify:

\begin{enumerate}
  \item Count how many switch positions are changed
  \item Count the number of atomic islands that have changed template
\end{enumerate}

Option 1 is very well suited as an effort measurement, i.e. how much effort 
it takes the grid operator to switch to that the new state. If $A_x = \{0, 1, ...\}$ 
is the set of switch states where $x$ is a partition and switch positions
are denoted as 1 for closed and 0 for open then

\begin{equation}
\Delta_{s} = |A_a \oplus A_b|
\end{equation}

expresses the number of changed switch positions.\\

Option 2 is well suited to quantify how significant in terms of grid topology 
the change is. If the number of templates stays constant between partitions then:

\begin{equation}
  \Delta_s = \sum_{i=0}^{|T_a|} ||T_{a, i}| - |T_{b, i}||
\end{equation}

\subsubsection{Degree}

The degree of a node is the number of edges connected to it. Using \autoref{eq:graph_theory:node_form}
the degree of a node can be written as:

\begin{equation}
  |\mathfrak{N_i}|
\end{equation}

The average degree
of partitions may be different if open switches are not considered. Similarly, 
if dead ends are disconnected it may change further.

\subsubsection{Density}

The density of a graph is defined as

\begin{equation}
  \frac{2 n_e}{n_n(n_n-1)} 
\end{equation}

It is both useful when applied to the full graph (with open switches removed)
or to individual templates. It can be applied both to the atomic island 
graph or the grid.

\subsubsection{Shortest path length}

The shortest path measures the shortest path between two nodes within a graph.
Particularly useful is the average shortest path length, which considers
the shortest path between any combination of nodes $i$ and $j$. This length can
either be considered as 1 for every edge used, the actual distance in meters or
as the impedance. Unfortunately
the metric is relatively expensive as it scales with the number of nodes squared.\\
An alternative is to take the average distance to a single node. In this case the
transformer should be a good choice.

\subsection{Powerflow based measures}



% - power stats (avg, std, etc)
% - voltage stats
% - line loading stats
% - trafo loading stats
% - Losses