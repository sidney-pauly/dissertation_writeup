\subsection{Analogy to physical annealing}

Simulated annealing is an optimization method that mimics the
physical process of cooling materials\autocite{simulated_annealing}.
Statistical mechanics teaches us that the probability of a system with
many particles to be in a specific state depends on its energy

\begin{equation}
    p(\{r_i\}) = e^{(-E(\{r_i\})/k_B T)},
\end{equation}

where $k_B$ is Boltzmann's constant, $T$ the temperature, $\{r_i\}$
the set of particle states and $E(\{r_i\})$ the energy of the entire state.
It can be seen that high energy states are more probable at higher temperatures.
However, at lower temperature these higher energy levels become more and more
unlikely. When a material is cooled
that has the ability to configure
into lower energy states it will do so as these lower energy states
are more probable. This process of cooling a material, often with a
specified cooling schedule, to make it settle into a specific type of configuration
is also called annealing, hence the name of the optimization method.\\
The process can be used for optimization by starting with a system at higher
energies and then "cooling" it, hoping for it to settle into a lower energy level.
A key advantage of this method is that it doesn't get stuck in local optima, 
other optimizers. This is because the probability for a higher energy state,
even though it is lower, it is not zero, especially at higher temperatures.

% \subsection{How this relates to switch states}

% To make simulated annealing work we have to model the switch state
% configurations like a statistical system. Firstly 

\subsection{Algorithm}

Simulated annealing as an algorithm works by exploring adjacent
configurations of a system and then assessing the probability of
the system going into the new state based on the energy
of the new and old state. A random draw is then made
utilizing that probability and if it comes out positive, the system
changes into that new state. 

\begin{algorithm}[H]
    
    \SetKwFunction{GetRandomAdjacentState}{getRandomAdjacentState} % Define the function name
    \SetKwFunction{UpdateTemperature}{updateTemperature}
    \SetKwFunction{SimulatedAnnealing}{simulatedAnnealing}
    \SetKwFunction{GetSystemEnergy}{getSystemEnergy}
    \SetKwFunction{P}{p}
    \SetKwFunction{Random}{random}

    \SetKwProg{Fn}{Function}{:}{} % Define the keyword and formatting
    \SetKw{KwContinue}{continue}

    \Fn{\SimulatedAnnealing{$state$, $minTemperature$}}{

        $T \gets 1$\;
        $E_{cur} \gets$ \GetSystemEnergy{$state$}\;

        \While{$T > minTemperature$}{

            $newState \gets$ \GetRandomAdjacentState{$state$}\;
            $E_{new} \gets$ \GetSystemEnergy{$newState$}\;

            $\Delta E \gets E_{cur}-E_{new}$\;

            \If{\P{$\Delta E$, $T$} $>$ \Random{}}{
                $state \gets newState$\;
                $E_{cur} \gets E_{new}$\; 
            }

            $T \gets $ \UpdateTemperature($T$)\;

        }

        \KwRet $state$\;
    }
    \caption{
        Simulated annealing algorithm, where \texttt{p()} is the probability of changing
        states (returns a probability between 0 and 1) and \texttt{random()} produces a
        random number between 0 and 1.
    }
    \label{alg:annealing}
\end{algorithm}

\subsection{Making it work for switch states}

\autoref{alg:annealing} show an implementation of the simulated annealing algorithm.
There are many ways to implement the probability function \texttt{p()}. For this work
a probability closely resembling the Boltzmann probability was chosen:

\begin{equation}
    p_c(\Delta E, T) = \begin{cases}
        1 & \text{if} \ \Delta E \leq 1\\
        e^{(-\gamma \Delta E/ T)} & \text{if} \ \Delta E > 1
    \end{cases} 
\end{equation}

where $\gamma$ is a constant that can be tweaked for best optimization performance.
For the energy of the system the scoring function (\autoref{eq:score}) is used:

\begin{equation}
    E = 1 - s
\end{equation}

Lastly to get a new random adjacent switch state \autoref{alg:ssexp:adjacent}
is used.
