% Mention that the faster we can run powerflow the more data can be collected later -> a main goal
% is to make powerflow as fast as possible
% Validation:
%   1. Comparision with results obtained by book
%   2. Comparision with vep results?
% Gauss-Seidel
% Newton-raphson
% python-c adapter
% sparse matricies optimization
% Show performance of different methods over grid size?

To solve the powerflow equation \ref{eq:pf:full_pf_eq} a numerical solver
is required\autocite{power_system_analysis}. In the context of this analysis
a vast number switch states have to be explored. The more swtichstates
can be explored in the same amount of time the better. Thus, a fast powerflow
solver is required. There are numerous solvers powerflow with different speed
and versatility traitoffs\autocite{pf_methods_comparison}. Some methods might
only work on specific grid topologies like radial topologies, some other methods
might require the network to have decoupled real and reactive power flows to work.
The Gauss-Seidel and Newton-Raphson methods are both comparatively robust and do not
have specific limitations on grid topology (meshed vs. radial)\autocite{pf_methods_comparison}.
Particularly the Newton-Raphson method is very fast for larger girds. Both methods
will be subsequently mathematically introduced and compared regarding their performance.
Subsequently, different implementations of the methods will be discussed and a performance
comparison between them and a widely used open source package \textit{Pandapower}\autocite{pandapower2018}
will be presented.

\subsection{Mathematical formulation}

\subsubsection{Gauss-Seidel}

The Gauss-Seidel method works by iteratively approximating the values of $x_i$ within a system of 
equations. Each subsequent approximation is then used in the next iteration. There is no guarantee that
the method converges and its convergence is highly dependent on the first guess being made. However the approach
works well for most powerflow problems\autocite{power_system_analysis}.\\
\\

To apply Gauss-Seidel the powerflow equation (\ref{eq:pf:full_pf_eq}) has to be re-arranged
by $V_i$ as such:

\begin{equation}
    V_i = \frac{\frac{S^*}{V_i^*} + \sum_{i \ne j}^N V_j y_{ij}}{y_{ii}}
\end{equation}

As can be seen the next guess for $V_i$ depends on the current value of $V_i$, we thus add a superscript $k$
to denote which iterative approximation of $V_i$ is meant:

\begin{equation}
    V_i^{(k+1)} = \frac{\frac{S^*}{{V_i^*}^{(k)}} + \sum_{i \ne j}^N V_j y_{ij}}{y_{ii}}
    \label{eq:pf:gauss_seidel}
\end{equation}

We can then apply a convergence criterion $\epsilon$ to be satisfied:

\begin{equation}
    |V_i^{(k+1)} - V_i^{(k)}| < \epsilon
    \label{eq:pf:convergance_criterion_gs}
\end{equation}

We keep applying eq. \ref{eq:pf:gauss_seidel} until \ref{eq:pf:convergance_criterion_gs}
is satisfied.

To speed up the algorithm further an acceleration factor $\alpha$ can be applied:

\begin{equation}
    V_{i \ acc}^{(k+1)} = V_i^{(k + 1)} + \alpha(V_i^{(k + 1)} - V_{i \ acc}^{(k)})
    \label{eq:pf:gs_pf}
\end{equation}

Typically, a value of $1.3 < \alpha < 1.7$ is recommeded\autocite{power_system_analysis}.

\subsubsection{Newton-Raphson}

Following \textit{Power System Analysis}\autocite{power_system_analysis} the Newton-Raphson method can be derived as follows. Starting with any equation like such:

\begin{equation}
    f(x) = c
    \label{eq:pf:nr_start_point}
\end{equation}

we can now decompose $x$ into an initial estimate $x^{(0)}$ 
and the difference $\Delta x^{(0)}$ of that estimate to the correct solution s.t.
$x = x^{(0)} + \Delta x^{(0)}$. This yields:

\begin{equation}
    f(x^{(0)} + \Delta x^{(0)}) = c
\end{equation}

Taylor expanding this equation yields

\begin{equation}
    f(x^{(0)}) + (\frac{df}{dx})^{(0)} \Delta x^{(0)} + \frac{1}{2!} (\frac{d^2f}{dx^2})^{(0)} (\Delta x^{(0)})^2 + ... = c
\end{equation}

Assuming a small error $\Delta x^{(0)}$ the higher may be dropped. Thus

\begin{equation}
    \Delta x^{(0)} \approxeq \frac{c - f(x^{(0)}) }{ (\frac{df}{dx})^{(0)} }
\end{equation}

This can now be used to obtain subsequent approximations:

\begin{align}
    x^{(1)}   &= x^{(0)} + \Delta x^{(0)}\\
    x^{(k+1)} &= x^{(k)} + \Delta x^{(k)}\\
    x^{(k+1)} &= x^{(k)} + \frac{c - f(x^{(k)}) }{ (\frac{df}{dx})^{(k)} }
    \label{eq:pf:nr_iteration}
\end{align}

\begin{figure}[H]
    \centering
    \begin{tikzpicture}[scale=2]

    \newcommand*{\eq}{=}
    \newcommand*{\interceptpos}{3.05}

    % Axis
    \draw[->,thick, line width=.7mm, name path=xaxis] (-.5,0) -- (4,0) coordinate(xlbl);
    \draw[->,thick, line width=.7mm] (0,-.5) -- (0,2.5);

    % fx
    \clip (-.7, -.7) rectangle (4, 2.5);
    \draw[
        -,
        blue,
        domain=-.5:4,
        smooth,
        line width=1mm,
        variable=\x,
        name path=fx
    ] plot ({\x},\x*\x*.2 -.3);

    \filldraw[
        red, 
        name intersections={of=fx and xaxis}
    ] (intersection-1) coordinate(x) circle (.5mm);

    \node at (x)[below] {$x$};

    % Tangent
    \draw[
        -,
        domain=-.5:4,
        smooth,
        line width=.9mm,
        variable=\x,
        black,
        name path=tangent
    ] plot({\x},{2*\interceptpos*.2*(\x - \interceptpos) + (\interceptpos*\interceptpos*.2 -.3)});

    \filldraw[
        red, 
        name intersections={of=tangent and xaxis}
    ] (intersection-1) coordinate(x1) circle (.5mm);

    \node at (x1)[below] {$x_1$};

    % Intercept
    \path[name path=intercept] (\interceptpos, 0) coordinate(x0) -- ++(0, 4);
    \draw[
        dotted, 
        line width=1mm,
        name intersections={of=fx and intercept}
    ] (x0) to [edge label= $\Delta y \eq f(x)$, sloped] (intersection-1) coordinate(v0);

    \filldraw [red] (x0) circle (.5mm);
    \filldraw [red] (v0) circle (.5mm);

    \node at (x0)[below] {$x_0$};
    \node at (v0)[left] {$f(x_0)$};

    \draw[<->, line width=1mm,name path=dx] ($(x1) - (0, .4)$) -- ($(x0) - (0, .4)$) node[pos=0.5](dxlbl){};

    \node at (dxlbl)[below]{$\Delta x$};

    % Formula

    \node[draw, fill=white] at (.5, 1) { $x_1 \eq x_0 - \frac{f(x_0)}{f'(x_0)}$ };


  \end{tikzpicture}
    \caption{
       Graphical representation of the Newton-Raphson method
    }
    \label{fig:pf:nr_graph}
\end{figure}

To make this approach work we need to arrange the powerflow equation \ref{eq:pf:full_pf_eq}
to the form \ref{eq:pf:nr_start_point}. We can choose between making $c = I$ or 
$c = S^*$:

\begin{align}
    \sum_{j = 1}^N V_i V_j y_{ij} = S^*_i\\
    \sum_{j = 1}^N V_j y_{ij} = I_i
    \label{eq:pf:nr_pf_starting_point}
\end{align}

thus $f(x)$ becomes:

\begin{align}
    f(x) &= S^*_i(V)\\
    f(x) &= I_i(V)
\end{align}

To formulate Newton-Raphson we also need the derivates $\frac{S^*_i(V)}{dV}$ or $\frac{I_i(V)}{dV}$. As $V$ and $S$ are vectors
(one value for each node $i$), the full derivative is a Jacobian:

\begin{align}
    \frac{S(V)}{dV} = J = 
    \begin{pmatrix}
        \frac{\partial S^*_1}{\partial V_1} & \frac{\partial S^*_1}{\partial V_2} & \dots  & \frac{\partial S^*_1}{\partial V_N}\\
        \frac{\partial S^*_2}{\partial V_1} & \ddots                              &        &                                    \\
        \vdots                              &                                     & \ddots &                                    \\
        \frac{\partial S^*_N}{\partial V_1} &                                     &        & \frac{\partial S^*_N}{\partial V_N}
    \end{pmatrix}
\end{align}

As these are complex number there are an additional three choices, the derivative can either be taken
with respect to the Cartesian from, Polar form or be left as a complex derivative.\\
\\
All these different choices do not only make a mathematical difference, they have impact
on computational and convergence performance\autocite{newton_raphson_setup_choices}. Complex formulations
generally perform the worst, whilst Polar and Cartesian formulations seem to have similar performance.
A current based setup performs slightly better in some scenarios\autocite{newton_raphson_setup_choices}.
However, due to the differences being minor and a wider adoption of the polar power formulation it will be used
here.\\

Reformulating eq. \ref{eq:pf:nr_pf_starting_point} in polar form yields the following:

\begin{equation}
    P_i = \sum_{j=1}^N |V_i||V_j||y_{ij}| \cos(\theta_{ij} - \delta_i + \delta_j)\\
    \label{eq:pf:nr_pf_p}
\end{equation}
\begin{equation}
    Q_i = -\sum_{j=1}^N |V_i||V_j||y_{ij}| \sin(\theta_{ij} - \delta_i + \delta_j)
    \label{eq:pf:nr_pf_q}
\end{equation}

To obtain the derivative we can again produce a jacobian:

\begin{equation}
    J =  
    \begin{bmatrix}
        \vec{P}(|\vec{V}|, \ \vec{\delta})\\
        \vec{Q}(|\vec{V}|, \ \vec{\delta})
    \end{bmatrix}
    \begin{pmatrix}
        \frac{\partial}{\partial |\vec{\delta}|} & \frac{\partial}{\partial |\vec{V}|}
    \end{pmatrix}
    =
    \begin{pmatrix}
        \frac{\partial \vec{P}}{\partial \vec{\delta}} & \frac{\partial \vec{P} }{\partial |\vec{V}|}\\
        \frac{\partial \vec{Q}}{\partial \vec{\delta}} & \frac{\partial \vec{Q} }{\partial |\vec{V}|}\\
    \end{pmatrix}
    =
    \begin{pmatrix}
        J_1 & J_2\\
        J_3 & J_4
    \end{pmatrix}
    \label{eq:pf:nr_pf_jacobi}
\end{equation}

Note that all the values above are vectors, thus each of the entries $J_n$ expand into
their own sub Jacobians.
This can be used to formulate \ref{eq:pf:nr_iteration} for powerflow:

\begin{equation}
    V^{(k+1)} = V^{(k)} + (P^{sch} - P^{(k)})J^{-1}
\end{equation}

where $P^{sch}$ is the power values given as input constraints, $P(k)$ the power
calculated for this iteration and $V$ and $P$ are row vectors as such:

\begin{equation}
    V = 
    \begin{pmatrix}
        \delta_1\\
        \dots\\
        \delta_N\\
        -\\
        \delta_1\\
        \dots\\
        \delta_N
    \end{pmatrix}
    \quad
    P = 
    \begin{pmatrix}
        P_1\\
        \dots\\
        P_N\\
        -\\
        Q_1\\
        \dots\\
        Q_N
    \end{pmatrix}
\end{equation}

This also gives us our convergence criteria in the form of:

\begin{align}
    \Delta P &= P^{sch} - P\\
    |\Delta P| &< \epsilon
    \label{eq:pf:nr_pf_convergance}
\end{align}

\begin{equation}
    V^{(k+1)} = V^{(k)} + \Delta P \ J^{-1}
\end{equation}

Again this formula is applied until \ref{eq:pf:nr_pf_convergance} is fulfilled.

\subsection{Implementation}

All powerflow solver code can be found in 
\texttt{lib/powerflow.py}, \texttt{lib/fastpf/powerflow.h}
\\ and
\texttt{lib/fastpf/powerflow.c}\\

The methods \texttt{solve\_powerflow\_gauss\_seidel()}
and \texttt{solve\_powerflow\_newton\_raphson()}
use python and numpy only and closely follow the mathematical formulations layed out above.\\

For the Gauss-Seidel method it is worth mentioning, that the Voltage result for each
node (\ref{eq:pf:gs_pf}) is immediately used to update the voltage vector. I.e. when $V_i^{(k+1)}$
is calculated the voltages $V_{j}^{(k+1)}$ where $j<i$ are already available and being used instead of
the old values $V_{j}^{(k)}$.\\

These methods where used as a baseline to access their performance. To identify where the biggest
speed-ups in performance could be gained timings where recorded for each of stages in the algorithm.
For a run of the python only Newton-Raphson timings to calculate the simbench grid shown in figure
\ref{fig:vep:simbench_node_types} are listed below:

\vspace{.5cm}

\begin{figure}[H]
    \begin{center}
        \begin{tabular}{ll}
            \textbf{Method section} & \textbf{Time (ms)}\\
            \hline
            Prepare matrices and vectors & 0.108\\
            Fill jacobian with zeros &  0.050\\
            Fill jacobian with values & 296.843\\
            Calculate Deltas &      0.067\\
            Invert Jacobian &      1.099\\
            Update voltage vector & 0.235\\
            \hline
            \textbf{Total} & \textbf{298.487}
        \end{tabular}
    \end{center}
\caption{Timings taken for different parts of the \texttt{solve\_powerflow\_newton\_raphson()} method}
\end{figure}
    
From this table it can immediately be seen, that the most important part
of the algorithm to optimize is filling the Jacobian with values. Thus, this
part was implemented in c using python c bindings to directly fill a numpy
matrix passed with values. This has the major advantage that only the most performance
critical code has to be implemented in c whilst everything else can be kept in python
for easy and quick development. Additionally, passing an existing matrix from python
means no additional memory allocations have to be done in c which is both good for performance 
and memory safety.\\
Taking the same timing for the c enhanced Newton-Raphson implementation yields the following:

\begin{figure}[H]
    \begin{center}
        \begin{tabular}{ll}
            \textbf{Method section} & \textbf{Time (ms)}\\
            \hline
            Prepare matrices and vectors & 0.232\\
            Fill jacobian with zeros &  0.051\\
            Fill jacobian with values & 0.436\\
            Calculate Deltas &      0.032\\
            Invert Jacobian &      1.493\\
            Update voltage vector & 0.016\\
            Finalize & 0.016\\
            \hline
            \textbf{Total} & \textbf{2.276}
        \end{tabular}
    \end{center}
\caption{Timings taken for different parts of the \texttt{fast\_newton\_raphson()} method}
\end{figure}

Using the new c method yields a considerable speed-up by more than ten times.\\

To unlock a further speed gain the matrices in use need to be examined closely. A normal admittance matrix has 
an entry for every possible combination of nodes $i$ and $j$, this means it has size $N^2$. However, if there is
no cable betewen $i$ and $j$ this entry is zero. As a lot of these cables do not exist a lot of the entries are zero
and the admittance matrix is sparse.
As can be seen from equation \ref{eq:pf:nr_pf_jacobi} as well as \ref{eq:pf:nr_pf_p} and \ref{eq:pf:nr_pf_q} the entries
in the Jacobian are also zero if a given entry $y_{ij}$ is zero in the admittance matrix.\\
Just iterating over these zero entries or doing multiplications of zero values wastes considerable processor time. On top
the large matrices take up a lot of memory. This problem can be solved by specifically using
a sparse matrix library like \textit{scipy.sparse}\autocite{2020SciPy-NMeth}. The library offers
data types that only store non-zero elements within the matrix as well as solvers that can operate
efficiently on these types.\\
Using the sparse library as well as implementing c code that operates on this matrix type a
further performance gain is possible:

\begin{figure}[H]
    \begin{center}
        \begin{tabular}{ll}
            \textbf{Method section} & \textbf{Time (ms)}\\
            \hline
            Prepare matrices and vectors & 0.075\\
            Fill power vec. with zeros &  0.007\\
            Fill jacobian with values & 0.179\\
            Calculate Deltas &      0.024\\
            Invert Jacobian &      0.452\\
            Update voltage vector & 0.015\\
            Finalize & 0.013\\
            \hline
            \textbf{Total} & \textbf{0.765}
        \end{tabular}
    \end{center}
\caption{Timings taken for different parts of the \texttt{fast\_newton\_raphson\_sparse()} method}
\end{figure}

\subsection{Validation}

\begin{figure}[H]
    \centering
    \begin{tikzpicture}[scale=1]


    % Slack
    \draw[line width=.8mm] (-4, -1) coordinate(s1) -- ++(0, 2.5) coordinate(e1) node(slack)[above]{1: Slack Node};
    \node (voltage_label) at ($(s1) - (.2, 0)$)[left] {$V_1 = 1.05 pu$};

    % Load 1
    \draw[line width=.8mm] (4, -1) coordinate(s2) -- ++(0, 2.5) coordinate(e2) node(slack)[above]{2: Prosumer};
    \node (q_label1) at ($(s2) + (.2, 0)$)[left][right] {$Q = 110.2Mvar$};
    \node (p_label1) at ([yshift=2mm]q_label1)[above] {$P = 256.6MW$};

    % Load 2
    \draw[line width=.8mm, name path=l2] (-1.5, -2) coordinate(s3) -- ++(3, 0) coordinate(e3);
    \node(slack) at ($(s3) - (0, .3)$)[left]{3: Prosmuer};
    \node(p_label2) at ($(e3) - (0, .3)$)[right]{$P = 138.6MW$};
    \node(q_label2) at ([yshift=-2mm]p_label2)[below]{$Q = 45.2 Mvar$};

    % Lines
    \draw[line width=.5mm] ($(e1) - (0, .7)$) -- ($(e2) - (0, .7)$) node[pos=.5](l1_midpoint){};
    \node(line_lbl_1) at (l1_midpoint)[above]{$y_{12} = 10-j20$};

    \draw[line width=.5mm] ($(s1) + (0, .7)$) -- ++(3, 0) coordinate(l2_end) node[pos=.5](l2_midpoint){};
    \node(line_lbl_2) at (l2_midpoint)[above]{$y_{13} = 10-j30$};

    \path[name path=intercept] (l2_end) -- ++(0, -4);
    \draw[
        line width=.5mm,
        name intersections={of=intercept and l2}
    ] (l2_end) to (intersection-1);

    \draw[line width=.5mm] ($(s2) + (0, .7)$) -- ++(-3, 0) coordinate(l3_end) node[pos=.5](l3_midpoint){};
    \node(line_lbl_3) at (l3_midpoint)[above]{$y_{23} = 16-j32$};

    \path[name path=intercept2] (l3_end) -- ++(0, -4);
    \draw[
        line width=.5mm,
        name intersections={of=intercept2 and l2}
    ] (l3_end) to (intersection-1);

  \end{tikzpicture}
    \caption{
        A simple 3 node system taken from \textit{Power System Analysis}\autocite{power_system_analysis}.
        Admittances $y_{ij}$ are in $pu$ (power unit) on a 100-MVA base.
        $P$ and $Q$ of prosumers are positive for consumption and negative for production.
    }
    \label{fig:pf:3_node_system}
\end{figure}


Fast powerflow is good, but no use if it is not correct. For validation
the solvers are compared against the system shown in figure \ref{fig:pf:3_node_system} and results
provided in \textit{Power System Analysis}\autocite{power_system_analysis}.

\begin{figure}[H]
    \begin{center}
        \begin{tabular}{m{3cm} | m{2.5cm} m{2.5cm} m{2.5cm} m{2.5cm}}
            Method & Slack real Power $P_1$ (MW) & Slack reactive Power $Q_1$ (Mvar) & Node 2 Voltage $V_2$ ($pu$) & Node 3 Voltage $V_2$ ($pu$)\\
            \hline
            Gauss-Seidel in \textit{Power System Analysis}\autocite{power_system_analysis} & $409.5$ & $189$ & $0.9800 - j0.0600$ & $1.0000 - j0.0500$\\
            \hline
            Own Gauss-Seidel implementation & 409.4 & 188.9 & $0.9800-j0.0600$ & $1.0000 - j0.0500$ \\
            \hline
            Own Newton-Raphson implementation (basic) & 409.5 & 189 & $0.9800-j0.0600$ & $1.0000 - j0.0500$ \\
        \end{tabular}
    \end{center}
    \caption{Results obtained through different methods for the grid shown in figure \ref{fig:pf:3_node_system}}
\end{figure}

The table above shows that the results obtained with our own Gauss-Seidel and Newton-Raphson
implementations agree with each other and with the values calculated in \textit{Power System Analysis}\autocite{power_system_analysis}. The accelerated
Newton-Raphson methods have also been compared with the values shown above, they are exactly equal to the values 
for the basic Newton-Raphson implementation

\subsection{Performance benchmarks}

