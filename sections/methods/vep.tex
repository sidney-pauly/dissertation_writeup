The \textbf{V}enios\textbf{E}nergy\textbf{P}latform is a cloud application
developed and hosted by Venios. The customers are grid operators who can use
it to visualize the grid, track real time data, have it predict 
operational parameters such as prosumer loads and control the grid.
The long term aim of this project is to develop an algorithm that takes in
congested grid areas as its input and advises VEP on possible improved
switch states and their effects. This dissertation aims to implement the first step of this
process by taking in grid data from VEP and analysing it, trying to figure out
improved grid configurations. Data obtained for this project from VEP can broadly
be categorized
into topological and load profiles. Whilst the former is anything having to do with
the grid structure, its geographical location and its operating parameters the later
are production or consumption values over time for each of the prosumers.\\
All data is obtained over Venios' API, in json format. This means any
grid area of any VEP customer can be used for this analysis just by changing out
the request parameters and use of the relevant authorization tokens for that customer.
No code changes are required. This approach has the added advantage that any customer grid
data is only stored in memory and only temporarily on the machine running the analysis. This
has data privacy advantages.

\subsection{Topological grid data}

The grid data coming from VEP contains edges and nodes.
The subtypes with relevance to this work are listed in tables \ref{table:vep:nodes} and
\ref{table:vep:edges}\\
\\
\begin{figure}[H]
    \begin{tabular}{l p{3cm} p{8cm}}
        Name & Power production/consumption & Description\\
        \hline
        Domestic prosumer & Consumes power & A household consuming power. Power is mainly consumed in the morning and evenings when people are home\\
        Industry prosumer & Consumes power & A store or factory. Power is mainly consumed during work hours\\
        Solar panel       & Produces power & Produces power based on sunlight, thus most power produced in the summer at mid-day\\
        Other prosumers   & Produces and consumes power & Other prosumers, like electric car chargers, batteries, heat pumps, etc. These can have a range of load patterns\\
        Any other node    & None           & Any other node acting as a junction, busbar or other. As they all do not consumer or produce power they are only relevant as so far as they act as junctions for cable connections
    \end{tabular}
    \caption{Node types relevant to this work}
    \label{table:vep:nodes}
\end{figure}

\begin{figure}[H]
    \begin{tabular}{l p{10cm}}
        Name & Description\\
        \hline
        Cable & Cables have impedance and a physical length\\
        Transformer & Connects a grid to a higher voltage level. Main feed-in point in a low voltage grid\\
        Switch & An edge that can be interrupted. Modelled with no impedance and no physical length\\
        Any other & Other miscellaneous edge not modelled with physical length or impedance. Usually connects nodes within a substation or electrical cabinet.
      \end{tabular}
    \caption{Edge types relevant to this work}
    \label{table:vep:edges}
\end{figure}

A good dataset for validation and comparison is the SimBench data set. It is a set of
gird data provided by the University Kassel. The dataset aims to model typical German
grids in rural and urban environments. It was specifically designed to include a higher
number of renewables in the low voltage level, like solar panels\autocite{simbench}. These
simbench grids can be also imported in VEP and where thus used as the initial validation case
for this work.\\

Figure \ref{fig:vep:simbench_node_types} shows one of these Simbench grids ("1-LV-rural1"). 
Within VEP this grid is modelled with 553 edges and 553 nodes. A lot of these nodes are there to model
internals like bus bars or protective switch setups. For power flow and for analysing
switch states these nodes can be merged down to 96 with 95 edges running between them. 
The exact method to do so will be explained in the data preparation section. These numbers of nodes
are typical for German grids\autocite{venios}.

\begin{figure}[h]
    \includegraphics{img/simbench/layout.png}
    \caption{
        Non-geographical representation of the SimBench "1-LV-rural1"
        grid\autocite{simbench}. Layout produced with
        Kamada-Kawai layout algorithm\autocite{kamada_kawai}.
        Nodes with connected transformers are shown
        in \textbf{grey}, households with solar panels in \textbf{green}
        and normal households in \textbf{red}. Only cables with non-zero impedance shown, any nodes
        connected by zero impedance cables have been merged.
    }
    \label{fig:vep:simbench_node_types}
\end{figure}

% The grouping into grids
% Outer world (for geographical locations)
%   Nodes
%   Edges
% Inner world (conducting)
%   Nodes (simple, prosumer, slack)
%   Edges (switches, simple, with impedance, transformers)
% Standard switchstate

\subsection{Grid Load}

To simulate the grid adequately, production and consumption values for
each of the prosumers is needed. VEP provides various different models
to do so ranging form standard load profiles (standardized curves which are adjusted
according to the prosumer's parameters), neural network models trained on historic data,
weather forecast based models, and actual real time measurement values\autocite{venios}.\\
\\
Within real low voltage girds, the data available will often be spotty. Not all prosumers
have measurement devices and their measurement frequency is often limited. The most accurate
load modelling data is therefore often a mixture of the sources mentioned above.\\
\\ 
For this work all load time series will be obtained from Venios through its API. Venios works
closely with the grid operators to make sure these timelines most closely resemble the conditions
in the real grid. This important as these values are being used to monitor the grid and in some
cases trigger active control measures\autocite{venios}. Within this work it is assumed that the data
available through the API is the best data available, which closely resembles the actual grid load.
Any switch states and their effects on grid parameters will be tested against this data. It can
thus be reasonably assumed that any improvements seen based on this data translate reasonably well
into the real world. 

\begin{figure}[H]
    \begin{subfigure}{.5\textwidth}
      \centering
      \includegraphics[width=\linewidth]{img/simbench/load_profile_household.png}
      \caption{}
      \label{fig:vep:load_profile_household}
    \end{subfigure}%
    \begin{subfigure}{.5\textwidth}
      \centering
      \includegraphics[width=\linewidth]{img/simbench/load_profile_solar.png}
      \caption{}
      \label{fig:vep:load_profile_solar}
    \end{subfigure}
    \caption{Power produced/consumed by a household (a) and a solar panel (b) within a SimBench grid at different times of day and at different times of year. Data is generated by a standard laod profile model within VEP and obtained over API\autocite{venios}}
    \label{fig:vep:load_profiles}
\end{figure}

Figure \ref{fig:vep:load_profiles} shows example load data obtained from VEP. It can be seen
that power production and consumption varies, widely throughout the day and the year. As a result the grid
might act very differently depending on the season and time. An important question to answer is therefore if
there are swtichstates that are generally better or if the best switch state is highly dependent on
time.\\
\\
Often switches within the grid can only be operated manually\autocite{venios}. Therefore, changing the grid configuration is linked to
some effort and cannot be done very often. If it turns out that there are switch states that are consistently better
a new and improved permanent switch state could be considered. If the best switch state is
however dependent on time, then depending on the scale of improvement, remotely controlled switches might be a worthwhile
investment.