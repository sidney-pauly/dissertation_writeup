The \textbf{V}enios\textbf{E}nergy\textbf{P}latform is a cloud application
developed and hosted by Venios. The customers are grid operators who can use
it to visualize the grid, track real time data, have it predict 
operational parameters such as prosumer loads and control the grid.
The long term aim of Venios is to have an algorithm that takes in
congested grid areas as its input and advises the platform on possible improved
grid states and their effects. This work aims to implement the first step of this
process by taking in grid data from VEP and analysing it, trying to figure out
improved grid configurations. Data obtained for this project from VEP can broadly
be categorized
into topological and load profiles. Whilst the former is anything having to do with
the grid structure, its geographical location and its operating parameters the later
are production or consumption values over time for each of the prosumers.\\
All data is obtained over Venios' API, in json format. This means any
grid area of any VEP customer can be used for this analysis just by changing out
the request parameters and use of the relevant authorization tokens for that customer.
No code changes are required. This approach has the added advantage that any customer grid
data is only stored in memory and only temporarily on the machine running the analysis. This
has data privacy advantages.

\subsection{Data exchanged}

Data elements in VEP are identified by their \texttt{ForeignKey} (abbrv. to \texttt{FK}). This is a unique key which
uniquely identifies the element and is sufficient to retrieve it. Data retrieved from the API might
be a combination of many data elements and might therefore not have a single \texttt{ForeignKey}, however
it might mention other elements by their \texttt{ForeignKey} so that they can be retrieved.

\subsubsection{API methods}

\begin{tabular}{ l  p{12cm}} 
    \hline
    \multicolumn{2}{c}{\textbf{Voltage groups}}\\
    \hline
    Method Name     & \texttt{GetAllVoltageGroups} \\
    Input           & -\\
    Output          & \texttt{list of VoltageLevel} \\
    Description     & Returns all voltage levels available in the current VEP instance. A voltage level is a level within the grid, normal households are usually connected to the LV (low voltage) voltage level at 400V\\
\end{tabular}

\vspace{.5cm}

\begin{tabular}{ l  p{12cm}} 
    \hline
    \multicolumn{2}{c}{\textbf{Grids in geographical boundary}}\\
    \hline
    Method Name     & \texttt{LoadGridsInGeoBounds} \\
    Input           & \texttt{VoltageLevel, GeoBounds}\\
    Output          & \texttt{list of GridFK}\\
    Description     & Returns foreign keys to grids within a geographical boundary and voltage level. A "grid" in vep is an administrative set of grid components usually connected to one transformer\\
\end{tabular}

\vspace{.5cm}

\begin{tabular}{ l  p{12cm}} 
    \hline
    \multicolumn{2}{c}{\textbf{Conducting topology}}\\
    \hline
    Method Name     & \texttt{LoadManyConductingTopologies} \\
    Input           & \texttt{list of GridFK}\\
    Output          & \texttt{VoltageLevel, GeoBounds} \\
    Description     & Returns the conducting topology of a grid, these are its cables, nodes and transformers\\
\end{tabular}

\vspace{.5cm}

\begin{tabular}{ l  p{12cm}} 
    \hline
    \multicolumn{2}{c}{\textbf{Get many}}\\
    \hline
    Method Name     & \texttt{GetMany} \\
    Input           & list of FK\\
    Output          & Raw VEP data element \\
    Description     & Used to retrieve a list of VEP data elements of any data type. Used here to retrieve transformer and cable specifications\\
\end{tabular}

\subsubsection{Complex data types}

Below is a description of the relevant complex data types retrieved from VEP. This is an incomplete
description only showing relevant data fields.




\vspace{1cm}



\subsection{Topological grid data}





% The grouping into grids
% Outer world (for geographical locations)
%   Nodes
%   Edges
% Inner world (conducting)
%   Nodes (simple, prosumer, slack)
%   Edges (switches, simple, with impedance, transformers)
% Standard switchstate

\subsection{Grid Load}

% Explain how data is imported and the format it comes in
% Data sources:
%   Standard load profile
%   Measurement based
%   Weather forcast based
%   Ai trained models

% Discuss how times of day or time of year influences the grid load
% and mention how that complicates the analysis