To compare switch states against each other a switch state score is required. 
Within the OTS and SAP literature a monetary cost is usually used, by either
calculating the cost of outages in the SAP\autocite{switch_allociation_lit_review} or by calculating the overall cost
of energy generation in the case of OTS\autocite{ots_lit_review}. The two papers
looking into OSSP or OSCP also use a cost function, here the cost of expansion
is considered, i.e. how much it would cost to add new prosumers to the grid
for different switch states.\\
\\
Directly optimizing for monetary cost is not desirable within the scope of this
project. Firstly, it is hard to estimate what costs are associated
to for example cable and transformer wear. Further there might even
be monetary conflict between short term costs and long term costs, e.g.
fewer line losses might be incurred by running the transformer at
a higher utilization, which would incur long term costs. \\
This project aims to develop a general switch state optimization
framework
applicable to all of Venios' customers. Different grid operators
in different regions, might pay different amounts for components, maintenance, electricity
or penalty charges.\\
Lastly this optimizer might be used in conjunction with different optimizers.
If a grid operator for example has to connect a new prosumer to their grid,
then different options might be weight depending on how they could somehow
accommodate the new prosumer. If the switch state optimizer is able
to return a result where the additional prosumer can be connected without
any grid expansion and other options do require grid expansion, then
using switching will likely always be cheaper\autocite{venios}. 
This is to highlight that an optimizer that optimizes for operational
grid parameters will be easier to accommodate then one that already optimizes
for cost.\\
\\
As there is no precedent for non-monetary cost functions for OSSP or OSCP adjecent
problems we developed our own grid score. The aim of this score
is to balance different operational parameters of the grid and unify them into a single number.
As different grid operators will have different requirements an additional aim
is to make the cost function easily fine tuneable by the grid operator themselves.
To achieve this the cost
function uses the operational grid quality parameters already known and
used by the grid operators and makes it possible to weigh them against each other.\\

\subsubsection{The score function}

The score is calculated as follows:

\begin{align}
    s &= s_{voltage} * w_{voltage}\\
      &+ s_{avg \ line} * w_{avg \ line} + s_{max \ line} * w_{max \ line}\\
      & + s_{avg \ trafo} * w_{avg \ trafo} + s_{max \ trafo} * w_{max \ trafo}\\
      & + s_{losses} * w_{losses},
\end{align}

where $s_x$ are the individual score elements and $w_x$ the weights
applied. The weights are normalized to sum to 1:

\begin{equation}
    w_{voltage} + w_{avg \ line} + w_{max \ line} + w_{avg \ trafo} + w_{max \ trafo} + w_{losses} = 1.
\end{equation}

All individual scores $s_x$ are also in the range from 0 to 1, s.t. the
entire score sums up to 1.\\
\\
The individual scores are calculated as such:

\begin{equation}
    s_{voltage} = 
\end{equation}