To compare switch states against each other a switch state score is required. 
Within the OTS and SAP literature a monetary cost is usually used, by either
calculating the cost of outages in the SAP\autocite{switch_allociation_lit_review} or by calculating the overall cost
of energy generation in the case of OTS\autocite{ots_lit_review}. The two papers
looking into OSSP or OSCP also use a cost function, here the cost of expansion
is considered, i.e. how much it would cost to add new prosumers to the grid
for different switch states.\\
\\
Directly optimizing for monetary cost is not desirable within the scope of this
project. Firstly, it is hard to estimate what costs are associated
to for example cable and transformer wear. Further there might even
be monetary conflict between short term costs and long term costs, e.g.
fewer line losses might be incurred by running the transformer at
a higher utilization, which would incur long term costs. \\
This project aims to develop a general switch state optimization
framework
applicable to all of Venios' customers. Different grid operators
in different regions, might pay different amounts for components, maintenance, electricity
or penalty charges.\\
Lastly this optimizer might be used in conjunction with different optimizers.
If a grid operator for example has to connect a new prosumer to their grid,
then different options might be weight depending on how they could somehow
accommodate the new prosumer. If the switch state optimizer is able
to return a result where the additional prosumer can be connected without
any grid expansion and other options do require grid expansion, then
using switching will likely always be cheaper\autocite{venios}. 
This is to highlight that an optimizer that optimizes for operational
grid parameters will be easier to accommodate then one that already optimizes
for cost.\\
\\
As there is no precedent for non-monetary cost functions for OSSP or OSCP adjecent
problems we developed our own grid score. The aim of this score
is to balance different operational parameters of the grid and unify them into a single number.
As different grid operators will have different requirements an additional aim
is to make the cost function easily fine tuneable by the grid operator themselves.
To achieve this the cost
function uses the operational grid quality parameters already known and
used by the grid operators and makes it possible to weigh them against each other.\\

\subsubsection{The scoring function}

The score is calculated as follows:

\begin{align}
    \begin{split}
        s &= s_{voltage} * w_{voltage}\\
        &+ s_{line \ avg} * w_{line \ avg} + s_{line \ max} * w_{line \ max}\\
        & + s_{trafo \ avg} * w_{trafo \ avg} + s_{trafo \ max} * w_{trafo \ max}\\
        & + s_{loss} * w_{loss},
    \end{split}
    \label{eq:score}
\end{align}

where $s_x$ are the individual score elements and $w_x$ the weights
applied. The weights are normalized to sum to 1:

\begin{equation}
    w_{voltage} + w_{line \ avg} + w_{line \ max} + w_{trafo \ avg} + w_{trafo \ max} + w_{losses} = 1.
\end{equation}

All individual scores $s_x$ are also in the range from 0 to 1, s.t. the
entire score sums up to 1.\\
\\

\subsubsection{Voltage score}

The voltage score is defined as:

\begin{equation}
    s_{voltage} = clamp(\frac{V_{rated}*(1+\Delta_{V}) - V_{max}
                +       V_{min} - V_{rated}*(1-\Delta_{V})}
                {{V_{rated}*\Delta_{V}}*2}),
                \label{eq:score:voltage}
\end{equation}

where $V_{rated}$ is the rated operating voltage of the grid, $\Delta_V$ the maximum allowed
voltage deviation as a fraction of the rated voltage, $V_{min}$ the minimum voltage  within
the switch state, $V_{max}$ the maximum voltage within the switch state and $clamp()$ being
defined as:

\begin{equation}
    clamp(x) =
    \begin{cases}
        0 & \text{if} \ x < 0\\
        x & \text{if} \ 0 < x < 1\\
        1 & \text{if} \ x > 1
    \end{cases}.
\end{equation}

\subsubsection{Line utilization score}

The maximum line utilization score is defined as

\begin{equation}
    s_{line \ avg} = clamp(\frac{\eta_{line \ max} - max(\{\eta_{ij} : (i, j) \in \mathfrak{E}_{ij}\})}{\eta_{line \ max}})
\end{equation}

average score as

\begin{equation}
    s_{line \ avg} = clamp(\frac{\eta_{line \ avg} - \frac{\sum_{(i, j) \in \mathfrak{E}_{ij}} \eta_{ij}}{|\mathfrak{E}_{ij}|}}{\eta_{line \ avg}}),
\end{equation}

where $\eta_{line \ max}$ and $\eta_{line \ avg}$ are the maximum allowed maximum or average utilization
values before the score drops to 0. $\eta_{ij}$ is used as defined in \autoref{eq:measures:cable_utilization}.
$\mathfrak{E}_{ij}$ is the set of all cables of the grid as defined in \autoref{eq:graph_theory:edge_list}
(Important: it is not just the set of edges of one template).

\subsubsection{Transformer utilization score}

The transformer utilization score is defined similarly to the cable utilization score with

\begin{equation}
    s_{trafo \ avg} = clamp(\frac{\eta_{trafo \ max} - max(\{\eta_{t} : 0 < t < |T|\})}{\eta_{trafo \ max}})
\end{equation}

average score as

\begin{equation}
    s_{trafo \ avg} = clamp(\frac{\eta_{trafo \ avg} - \frac{\sum_{t = 0}^{|T|} \eta_t}{|T|}}{\eta_{trafo \ avg}}),
\end{equation}

where $T$ is the set of templates and $\eta_{trafo \ max}$ as well as $\eta_{trafo \ avg}$ work similarly
as for the cable score.

\subsubsection{Line loss Score}

To calculate a line loss score that is comparable between switch states a suitable
normalization method needs to be employed. The approach taken by us is to use the 
sum over all generated and consumed power as such:

\begin{equation}
    |S|_{total} = \sum_{i=0}^N |P_i| + |Q_i|.
\end{equation}

The rational here being, that the more power is generated or consumed within a grid
the more losses should also occur. The absolute value is calculated are component wise
and not as a magnitude as resistive and reactive power losses also add up component wise.\\
\\
The line loss score is then defined as such:

\begin{equation}
    s_{loss} = clamp(\frac{\alpha \sum_{(i, j) \in \mathfrak{E}_{ij}} P_{ij, \ loss} + Q_{ij, \ loss}}{|S|_{total}}),
\end{equation}

where $\alpha$ is a correction term as line losses are usually an order of magnitude
smaller than $|S|_{total}$.

\subsubsection{Limits and weights used}

\begin{figure}[H]
    \begin{tabular}{l l p{10cm}}
        Limit or weight & Value & Notes\\
        \hline
        $V_{rated}$         & $400V$ & Voltage within the 3-phase low voltage grid\\
        $\Delta_V$          & $0.15$ & $15\%$ voltage deviation is often used as an acceptable limit by grid operators\autocite{venios}\\
        $\eta_{line \ avg}$ & $1$    & \\
        $\eta_{line \ max}$ & $2.5$  & Rated current is not maximum allowed current, as such higher currents may occur\\
        $\eta_{trafo \ avg}$& $1$    & \\
        $\eta_{trafo \ max}$& $1.5$  & Rated power is not maximum allowed power, as such higher currents may occur\\
        $\alpha$            & $40$   & Determined through trial and error\\
        $w_{voltage}$       & $1/4$  & All 4 score categories set to contribute equally\\
        $w_{line \ avg}$    & $1/8$  & All 4 score categories set to contribute equally, there are 2 line utilization measures\\
        $w_{line \ max}$    & $1/8$  & All 4 score categories set to contribute equally, there are 2 line utilization measures\\
        $w_{trafo\ avg}$    & $1/8$  & All 4 score categories set to contribute equally, there are 2 transformer utilization measures\\
        $w_{trafo\ max}$    & $1/8$  & All 4 score categories set to contribute equally, there are 2 transformer utilization measures\\
        $w_{loss}$          & $1/4$  & All 4 score categories set to contribute equally
    \end{tabular}
    \caption{Values used for the score function \autoref{eq:score}}
    \label{table:score:values}
\end{figure}

As mentioned all weights and limits are chosen such that they are meaningful to grid operators
and can be tweaked by them as needed. For all switch state scores within this work the values
in \autoref{table:score:values} have been used.
