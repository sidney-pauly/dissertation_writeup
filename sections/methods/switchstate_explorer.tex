Within the distribution grid there are two requirements for a switch
state:

\begin{enumerate}
    \item All prosumers are connected to a grid island
    \item Each grid island is connected to at least one transformer
\end{enumerate}

to limit the search space we restrict this further by imposing that each
island must be connected to exactly one transformer.\\
\\
More formally we can define a valid switch state
as some combination of open and closed switches,
that results in sets of adjacent nodes that have all switches
between them closed and all switches to other
sets opened. Further, it is a requirement that each of
these sets contains exactly one transformer.\\
\\
Within the literature this problem is known as graph partitioning, which is a form
of graph covering. 
The specific case needing to be solved here being anchored or 
rooted graph partitioning\autocite{graph_partitioning}. The anchors
or roots being the transformers.\\
\\
Knauer and Ueckerdt\autocite{graph_covering_terminology}
introduce the following terminology for the graph covering problem:
if there is a graph G (called
the host graph) then the set of possible sub-graphs is called a template class.
Elements of the template class are then called template graphs. A configuration 
where each node belongs to at least
one of these templates is called a covering. A quality metric can then be applied to a covering
that grades how well the covering is performing in a certain aspect. For some of these metrics
there exist optimizers that can reliably produce the covering with the best possible result for that
specific quality.\\
Graph partitioning is the same, with one additional constraint, that each node can only belong
to exactly one template. Such a covering is called a partition.\\
Finding switch states is finding graph partitions as each node should only
be contained in exactly one template (be part of one sub-graph) and contain at least
one transformer (one root), making it a rooted partitioning problem.\\
\\
Even for small numbers of nodes there are a huge number of possible partitions
of the grid.
The number of partitions will be referred to as $n_p$.
Just counting how many partitions exist is a non-trivial exercise.
As an upper bound it is safe
to assume that

\begin{equation}
    n_p < 2^{n_s},
    \label{eq:sw_exp:upper_bound1}
\end{equation}

where $n_{s}$ are the number
of switches. This is because each switch can either be opened or closed. However,
a lot of these switch configurations would not result in valid a partition.\\
An approach that could be taken is to generate switch states until a valid
partition is produced. However, this approach is not desirables as
there are a lot more invalid switch states than valid ones. As this
is computationally expensive a better approach is desirable.
