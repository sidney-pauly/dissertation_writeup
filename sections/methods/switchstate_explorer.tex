With the atomic islands determined we can now generate new swtichstates.
A switch state will be described as a combination of adjacent atomic
islands. Further we want all the islands to be connected to 
exactly one transformer. Having more than one transformer is possible, however to
restrict the search space, only configurations with one feeding transformer
will be considered initially.\\
\\
Within the literature this problem is known as graph partitioning, specifically
as anchored or rooted graph partitioning\autocite{graph_partitioning}. The anchors
or roots being the transformers. Knauer and Ueckerdt\autocite{graph_covering_terminology}
introduce the following terminology for the graph covering problem: if there is a garp G (called
the host graph) then
the set of possible sub-graphs is called a template class. Elements of the 
template class are then called template graphs. Each node or edge belongs to at least
one of these templates (this is called a covering). A quality metric can then be applied to a covering.
Graph partitioning is the same, with one additional constraint, that each node can only belong
to exactly one template. A switch state is nothing else then a template in the is framework, whilst
the host-graph is the graph of atomic islands. It is a partitioning problem as each node should only
be contained in exactly one template (be part of one grid island) and contain at least
one transformer, which act as the roots, making it a rooted partitioning problem.\\
\\
Within OTS the problem a similar problem exists: here cables
are to be taken offline. I.e. they start with all switches closed and then open
individual switches. The objective is generally to avoid disconnecting any
sections of the grid at all. I.e. there should only be one big island. This is different
from the situation in distribution grids: here it is ok to have different islands if they
are connected to at least one transformer.\\
\\
As the presented here there are huge number of possible partitons of the grid.
The number of partitions will be refered to as $n_p$.
Just counting them is a non-trivial exercise. As an upper bound is safe
to assume that

\begin{equation}
    n_p < 2^n_s 
    \label{eq:sw_exp:upper_bound1}
\end{equation}

where $n_{s}$ are the number
of switches. This is because each switch can either be opened or closed. However
a lot of these switch configuration would lead to coverings where not all nodes
are connected to at least one transformer. However the number of total switch
configurations is still very large. This means testing them all is not feasible.\\
\\
A good starting point is to generate random switch states, to get an overview of what the search
space looks like. In order to do this we implemented a custom flooding algorithm. The goal
of the algorithm is to get grid states with exactly one transformer per grid island.
In order to do this the 